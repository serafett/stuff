\chapter{Serialization}
Till so far the examples have relied on standard Java serialization by letting the objects we store in Hazelcast, implement the java.io.Serializable interface. But Hazelcast has a very advanced serialization systems that supports native java serialization mechanisms like Serializable and Externalizable;  useful if you don't own the class and therefor can't change its serialization mechanism. But it also supports a 2 custom serialization mechanisms: DataSerializable and Portable.

Serialization in Hazelcast works like this: when an object is placed in a Hazelcast data-structure, e.g. in a map or queue, Hazelcast first checks if it is an instance of DataSerializable or Portable. If that fails it checks if is a well known type like String, Long, Integer, byte[], ByteBuffer, Date etc, since serialization for these types can be optimized. Then user specified types [reference to TypeSerializers] If that fails it will fall back on Java serialization (including the Externalizable) or fails because the class can't be serialized. These sequence of steps also is useful to determine which serialization mechanism is going to be used by Hazelcast if a class implements multiple interfaces, e.g. Serializable and Portable.

\section{Serializable}
The native Java serialization is the easiest serialization mechanism since often a class only needs to implement the java.io.Serializable interface:
\begin{lstlisting}[language=java]
public class Person implements Serializable {
    private static final long serialVersionUID = 1L;
    private String name;
    public Person(String name) {
        this.name = name;}}
\end{lstlisting}
When this class is serialized, all non static non transient fields will automatically be serialized. Also make sure that a serialVersionUID is added since this prevents the JVM from calculating one on the fly which can lead to all kinds of class compatibility issues. In the examples they are not added to reduce space, but for production code there is no excuse.

\section{Externalizable}
Another serialization technique supported by Hazelcast is the java.io.Externalizable which provides more control on how the fields are serialized/deserialized and can also help to improve performance compared to standard Java serialization. An example of the Externalizable in action:
\begin{lstlisting}[language=java]
public class Person implements Externalizable {
    private String name;
    public Person(String name) {
        this.name = name;}
    public void readExternal(ObjectInput in) throws IOException, ClassNotFoundException {
        this.name = in.readUTF();}
    public void writeExternal(ObjectOutput out) throws IOException {
       out.writeUTF(name);}}
\end{lstlisting}
As you can see the writing and reading of fields is explicit. Unlike the Serializable, the serialVersionUID is not required.

\section{DataSerializable}
Although Java serialization is very easy to use, it comes at a price:
\begin{enumerate}
\item lack of control on how the fields are serialized/deserialized.
\item suboptimal performance due to streaming class descriptors, versions, keeping track of seen objects to deal with cycles etc. This causes additional cpu load and suboptimal size of serialized data.
\end{enumerate}
That is why in Hazelcast 1 custom serialization mechanism was introduced: DataSerializable, example:
\begin{lstlisting}[language=java]
public class Person implements DataSerializable {
    private String name;
    public Person(String name) {
        this.name = name;}
    public void readData(ObjectDataInput in) throws IOException {
        this.name = in.readUTF();}
    public void writeData(ObjectDataOutput out) throws IOException {
        out.writeUTF(name);}}
\end{lstlisting}
As you can see it looks a lot like the Externalizable functionality since also an explicit serialization of the fields is done. 

\section{Portable}
With the introduction of Hazelcast 3.0 a new serialization mechanism has been added: the Portable. To demonstrate how Portable mechanism works, lets create a Portable version of the Person class:
\begin{lstlisting}[language=java]
public class Person implements Portable {
    private String name;
    Person(){}
    public Person(String name) {
        this.name = name;}
    public int getClassId() {
        return PortableFactoryImpl.PERSON_CLASS_ID;}
    public void writePortable(PortableWriter writer) throws IOException {
        System.out.println("Serialize");
        writer.writeUTF("name", name);}
    public void readPortable(PortableReader reader) throws IOException {
        System.out.println("Deserialize");
        this.name = reader.readUTF("name");}
    public String toString() {
        return String.format("Person(name=%s)",name);}}
\end{lstlisting}
As you can see, the write of the name include the field name, making it possible to read particular fields without being forced to read all.  Unlike the DataSerializable, the order in which fields are being read/written isn't important since reading/writing is based  on name. The last interesting method is the 'getClassId' which returns the identifier of that class; each Portable class needs to have a unique class id (a non zero value).

The next step is the 'PortableFactory' which is responsible for creating a new Portable instance based on the class id. In our case the implementation is very simple since we only have a single Portable class:
\begin{lstlisting}[language=java]
import com.hazelcast.nio.serialization.*;
public class PortableFactoryImpl implements PortableFactory {
    public final static int PERSON_CLASS_ID = 1;
    public Portable create(int classId) {
        switch (classId) {
            case PERSON_CLASS_ID: return new Person();
        }
        return new [TODO: Exception]}}
\end{lstlisting}
But in practice the switch case probably will be a lot bigger. If an unmatched classId is encountered, an exception should be thrown [todo: which exception]. The cool thing about the PortableFactory is that object creation is pulled into userspace, so you control the initialization of the Portable instances. For example you could inject dependencies or you could even decide to move the construction of the Portable to a prototype bean in a Spring container. The last step is to connect the PortableFactory to the Hazelcast configuration:
\begin{lstlisting}[language=xml]
<serialization>
    <portable-factory-class>PortableFactoryImpl</portable-factory-class>
</serialization>
\end{lstlisting}
Of course we also want to see it in action:
\begin{lstlisting}[language=java]
public class Member {
    public static void main(String[] args) {
        HazelcastInstance hzInstance = Hazelcast.newHazelcastInstance();
        Map<String, Person> map = hzInstance.getMap("map");
        map.put("Peter", new Person("Peter"));
        System.out.println(map.get("Peter"));}}
\end{lstlisting}
When we run this PortableMember, we'll see the following output:
\begin{lstlisting}[language=java]
Serialize
Serialize
Deserialize
Person(name=Peter)
\end{lstlisting}
As you can see the Person is serialized when it is stored in the map and it is deserialized when it is read. You might ask why Serialize is called twice. This is because for every Portable class that is serialized, for the first time, Hazelcast generates a new class that supports the serialization/deserialization process. 

The names of the fields are case sensitive and need to be valid java identifiers and therefor should not contains '.' or '-' for example. 

\subsection{DataSerializable vs Portable}
Portable supports versioning and is language/platform independent which makes it useful for client/cluster communication. Another advantage is that is very performant for map queries, since it avoids full serialization because data can be retrieved on the field level. Otherwise, if the serialization only is needed for intra cluster communication, then DataSerializable is still a good alternative.

\subsection*{Object traversal}
If a Portable, has a Portable field, the write and read operations need to be forwarded to that object. For example if we would add a Portable address field to Person:
\begin{lstlisting}[language=java]
    public void writePortable(PortableWriter writer) throws IOException {
        writer.writeUTF("name", name);
        writer.writePortable("address", address);}
    public void readPortable(PortableReader reader) throws IOException {
        this.name = reader.readUTF("name");
        this.address = reader.readPortable("address");}
\end{lstlisting}

If the object is not a Portable, primitive, array or String then there is no direct support for serialization. Of course, you could transform the object using Java serialization to a byte array, but this would mean that the platform independence is lost. A better solution  is to create some form of String representation, potentially using XML, so that platform compatibility is maintained.

If the object is of type String, the writeUTF/readUTF work fine. 

[todo: what about object wrappers? I guess that there are some serializers registered for these types and the readPortable method can be used]
[todo: dealing with null, currently a NPE is thrown]

\subsection*{Serialize DistributedObject}
Serialization of the DistributedObject is not provided out of the box, so you can't put for example an ISemaphore on an IQueue on one machine, and take it from another. But there are solutions to this problem.

One solution is to pass the id of the DistributedObject, perhaps in combination with the type. When deserializing look up the object in the HazelcastInstance; e.g. in case of a IQueue you can call HazelcastInstance.getQueue(id) or the Hazelcast.getDistributedObject. Passing the type is useful if you don't know the type of the DistributedObject.

If you are deserializing your own Portable Distributed Object object and it receives an id that needs to be looked up; the class can implement the HazelcastInstanceAware interface, but this will be called after the deserialization. In that case it is best to store the id when deserializing and call the getDistributedObject when the setHazelcastInstance method is called.

\subsection*{Cycles}
One thing to look out for, this also goes for the DataSerializable, are cycles between objects because it can lead to a stack overflow. Standard Java serialization protects against this, but since manual traversal is done in Portable objects, there is no out of the box protection. If this is an issue, you could store a map in a ThreadLocal that can be used to detect cycles and a special placeholder value could be serialized to end the cycle.

\subsection*{Subtyping}
Subtyping with the Portable functionality is easy, let every subclass have its own unique type id and add these id's to the switch/case in the PortableFactory so that the correct class can be instantiated. 

\subsection*{Versioning}
In practice it can happen that multiple versions of same class are serialized and deserialized; imagine a Hazelcast client with an older Person class compared to the cluster. Luckily the Portable functionality supports versioning. In the configuration you can explicitly pass a version using the '<portable-version>' tag (defaults to 0):
\begin{lstlisting}[language=xml]
<serialization>
    <portable-factory-class>PortableFactoryImpl</portable-factory-class>
    <portable-version>1</portable-version>
</serialization>
\end{lstlisting}
When an Transportable instance is deserialized, next to the deserialized fields of that Portable also the class id and the portable-version will be stored. So every time you make a change to the class, the version should be incremented. 

Adding fields to a class is simple, removing fields can lead to problem if a new version of that class (with the removed field) is deserialized on a client which depends on that field. Although using the Portable interface is more work because of explicit mapping fields, it suffers less from field renaming, since name of the field in the class can be named independently of field name in the class. Another issue to watch out for is changing the field type, although although Hazelcast can do some basic type upgrading (e.g. int to long or float to double). 

Luckily Hazelcast provides access to the metadata of the to deserialize object through the PortableReader; the  version, available fields, the type of the fields etc can be retrieved. So you have full control on how the deserialization should take place. 

\section{TypeSerializers}
In Hazelcast 3.0, users can register a TypeSerializer for a specific class, superclass or interface using TypeSerializerConfig. For example user can register its own serializer or hook it up to a JSON, protobuf, Kryo etc. So lets create a TypeSerializer for a Person class:
\begin{lstlisting}[language=java]
public class Person {
    private String name;
    public Person(String name) {this.name = name;}}
\end{lstlisting}
The TypeSerializer for the Person class would look like this:
\begin{lstlisting}[language=java]
public class PersonTypeSerializer implements TypeSerializer<Person> {
    public void destroy() {}
    public int getTypeId() {
        return 1;}
    public void write(ObjectDataOutput out, Person object) throws IOException {
        out.writeUTF(object.getName());}
    public Person read(ObjectDataInput in) throws IOException {
        return new Person(in.readUTF());}}
\end{lstlisting}
On thing worth mentioning is that the type id needs to be unique. And the PersonTypeSerializer needs to be wired up in Hazelcast like this:
\begin{lstlisting}[language=xml]
<serialization>
    <serializers>
        <type-serializer type-class="Person">PersonTypeSerializer</type-serializer>
    </serializers>
</serialization>
\end{lstlisting}
As you can see, we have hooked up the PersonTypeSerializer to the Person class. So if Hazelcast runs into an instance of Person which needs to be (de)serialized, the PersonTypeSerialized will be called. It isn't possible to override all TypeSerializers since Hazelcast already registered TypeSerializers for a lot of well known types like primitives, primitive arrays, String, Date, etc.

\subsection{Global Serializer}
The new Hazelcast serialization functionality also makes it possible to configure a global serializer in case there are no other serializers found:
\begin{lstlisting}[language=xml]
<serialization>
    <serializers>
        <global-serializer>PersonTypeSerializer</global-serializer>
    </serializers>
</serialization>
\end{lstlisting}
There can only be a single global serializer. 
[todo: does the getTypeId still matter?]

\section{Good to know}

\emph{Caution} Unlike the DataSerializable, the Portable functionality is allowed to call operations on Hazelcast that leads to new serialize/deserialize operations and doesn't run into StackOverflowErrors. [todo: is this still valid?]

\section{What is next}
In this chapter we have seen different forms of serialization; making serialization extremely flexible, especially with the Portable and the TypeSerializers. In most cases this will be more than sufficient. But if you ever run into a limitation you could, create a task in github and perhaps it will be added to the next Hazelcast release.