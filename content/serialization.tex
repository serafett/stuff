\chapter{Different serialization mechanisms}
TODO: COPY PASTE
All your distributed objects such as your key and value objects, objects you offer into distributed queue and your distributed callable/runnable objects have to beSerializable.

Hazelcast serializes all your objects into an instance of com.hazelcast.nio.Data. Data is the binary representation of an object. When Hazelcast serializes an object intoData, it first checks whether the object is an instance of well-known, optimizable object such asString, Long, Integer, byte[], ByteBuffer, Date. If not, it then checks whether the object is an instance of com.hazelcast.nio.DataSerializable. If not, Hazelcast uses standard java serialization to convert the object into binary format. See com.hazelcast.nio.Serializer for details.

So for faster serialization, Hazelcast recommends to use of String, Long, Integer, byte[] objects or to implement com.hazelcast.nio.DataSerializable interface. Here is an example of a class implementing com.hazelcast.nio.DataSerializable interface.

\begin{lstlisting}[language=java]
public class Address implements com.hazelcast.nio.DataSerializable {
    private String street;
    private int zipCode;
    private String city;
    private String state;
    public Address() {}
    //getters setters..
    public void writeData(DataOutput out) throws IOException {
        out.writeUTF(street);
        out.writeInt(zipCode);
        out.writeUTF(city);
        out.writeUTF(state);
    }
    public void readData (DataInput in) throws IOException {
        street    = in.readUTF();
        zipCode = in.readInt();
        city    = in.readUTF();
        state    = in.readUTF();
    }
}
\end{lstlisting}

Lets take a look at another example which is encapsulating a DataSerializable field.

\begin{lstlisting}[language=java]
public class Employee implements com.hazelcast.nio.DataSerializable {
    private String firstName;
    private String lastName;
    private int age;
    private double salary;
    private Address address; //address itself is DataSerializable
    public Employee() {}
    //getters setters..
    public void writeData(DataOutput out) throws IOException {
        out.writeUTF(firstName);
        out.writeUTF(lastName);
        out.writeInt(age);
        out.writeDouble (salary);
        address.writeData (out);
    }
    public void readData (DataInput in) throws IOException {
        firstName = in.readUTF();
        lastName  = in.readUTF();
        age       = in.readInt();
        salary       = in.readDouble();
        address   = new Address();
        // since Address is DataSerializable let it read its own internal state
        address.readData (in);
    }
}
\end{lstlisting}
As you can see, since address field itself isDataSerializable, it is calling address.writeData(out) when writing and address.readData(in) when reading.

Caution: Hazelcast serialization is done on the user thread and it assumes that there will be only one object serialization at a time. So putting any Hazelcast operation that will require to serialize anything else will break the serialization. For Example: Putting

Hazelcast.getMap("anyMap").put("key", "dummy value");
line in readData or writeData methods will break the serialization. If you have to perform such an operation, at least it should be performed in another thread which will force the serialization to take on different thread.