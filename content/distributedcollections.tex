\chapter{Distributed Collections}
Hazelcast provides a set of collections that implement interfaces from the Java collection framework and therefor make it easy to integration distributed collections without too many code changes. This chapter will example distributed collections like:
\begin{enumerate}
\item BlockingQueue
\item Set
\item List
\end{enumerate}
The distributed map functionality provided by Hazelcast is so extensive that 2 separate chapters have been dedicated to it. 

One of the cool thing about distributed collections in Hazelcast is that when a member fails that manages a part of a distributed collection, another member will immediately take over without any elements in the collection getting lost.

\section{Distributed Queue}
A BlockingQueue is one of the working horses for concurrent system because it allows producers and consumers of messages (POJO's) to work different speeds. The Hazelcast java.util.concurrent.BlockingQueue implementation not only allows threads from the same member to interact with that queue, but since the queue is distributed, it also allows different members to interact with it concurrently.

As an example we'll create a producer/consumer implementation that is connected by a distributed BlockingQueue. The producer is going to put a 1000 messages on the queue (an integer) every second:
\begin{lstlisting}[language=java]
import com.hazelcast.
import java.util.concurrent.BlockingQueue;
public class ProducerMember {
    public static void main(String[] args) throws Exception {
        HazelcastInstance hzInstance = Hazelcast.newHazelcastInstance(null);
        BlockingQueue<Integer> queue = hzInstance.getQueue("queue");
        for (int k = 1; k < 1000; k++) {
            queue.put(k);
            System.out.println("Producing: " + k);
            Thread.sleep(1000);
        }
        queue.put(-1);
        System.out.println("Producer Finished!");
    }
}
\end{lstlisting}
To make sure that the consumers are going to terminate when the producer is finished, the producer will put a -1 on the queue to indicate that it is ready and when a consumer reads this message, it will terminate. Such a special control message also is called a poison pill.

The consumer will take the message from the queue, print it and waits 5 seconds before retrying and stops when it receives the poison pill:
\begin{lstlisting}[language=java]
import com.hazelcast.core.*;
import java.util.concurrent.BlockingQueue;
public class ConsumerMember {
    public static void main(String[] args) throws Exception {
        HazelcastInstance hzInstance = Hazelcast.newHazelcastInstance(null);
        BlockingQueue<Integer> queue = hzInstance.getQueue("queue");
        while (true){
            int item = queue.take();
            System.out.println("Consumed: " + item);
            if(item == -1){
                queue.put(-1);
                break;
            }     
            Thread.sleep(5000);            
        }
        System.out.println("Consumer Finished!");
    }
}
\end{lstlisting}
If you take a closer look at the consumer, you see that when the consumer receives the poison pill, it puts the poison pill back on the queue before it ends the loop. This is done to make sure that all consumer will receive the poison pill, and not only the one that received it first.

When you begin with starting a single producer, you will see the following output:
\begin{lstlisting}
Produced 1
Produced 2
Produced 3
....
\end{lstlisting}
When you start a single consumer, you will see the following output:
\begin{lstlisting}
Consumed 1
Consumed 2
Consumed 3
....
\end{lstlisting}
As you can see, the items produced on the queue by the producer are being consumed from that same queue by the consumer. 

Because messages are produced 5 times faster than they are consumed, with a single member the queue will keep growing. To improve scalability, you can start more consumers. If we start another consumer, we'll see that one consumer takes care of on half of the messages and the other consumer takes care of the other half.

Consumer 1:
\begin{lstlisting}
Consumed 20
Consumed 22
Consumed 24
....
\end{lstlisting}
Consumer 2:
\begin{lstlisting}
Consumed 21
Consumed 23
Consumed 25
....
\end{lstlisting}
And when you kill one of the consumers, the remaining consumer will process all elements again:

Consumer 1:
\begin{lstlisting}
Consumed 40  
Consumed 42 
Consumed 44 
Consumed 45
Consumed 46
....
\end{lstlisting}

TODO: Needs to be verified.
One thing to take care of that if there are many producers/consumers interacting with the queue, is that the queue eventually will become a bottleneck and this is caused by contention. One way of solving this problem is to introduce a stripe (essentially a list) of BlockingQueues. But if you do, the ordering of messages send to different queues will not be guaranteed anymore. In a lot of cases a strict ordering isn't required and a stripe can be a simple solution to deal with scalability.

\emph{Important}: Realize that although the Hazelcast distributed queue preserves ordering of the messages (so the messages are taken from the queue in the same order they were put on the queue), if there are multiple consumers, the processing order is not guaranteed. The queue will not provide any ordering guarantees on messages after they are taken.

\subsection{BlockingQueue Capacity}
In the previous example we created a basic producer/consumer solution based on a distributed queue. Because the production of messages is separated from the consumption of messages, the speed of production is not influenced by the speed of consumption. If producing messages goes quicker than the consumption, then the queue will increase in size. If there is no bound on the capacity of the queue, machines can run out of memory and you will get an OutOfMemoryError. 

With the traditional BlockingQueue implementations like the LinkedBlockingQueue, a capacity can be set. When this is set and the maximum capacity is reached, placement of new items either fail or block, depending on type of the put operation. This prevents the queue from growing beyond a healthy capacity and the JVM from failing.

Hazelcast also provided control in the capacity, but instead of having a fixed capacity for the whole cluster, Hazelcast provides a scalable capacity by setting the queue capacity per member. So if the capacity per member is 1000 and there are 5 members's, the total capacity is 5000. So capacity depends on the size of the cluster.

To give our queue a capacity of 10 per member, we add the following to our hazelcast.xml file:
\begin{lstlisting}
<hazelcast xsi:schemaLocation="http://www.hazelcast.com/schema/config hazelcast-config-2.3.xsd"
           xmlns="http://www.hazelcast.com/schema/config"
           xmlns:xsi="http://www.w3.org/2001/XMLSchema-instance">
    <network>
        <join>
            <multicast enabled="true"/>
        </join>
   </network>
    <queue name="queue">
        <max-size-per-jvm>10</max-size-per-jvm>
    </queue>
</hazelcast>
\end{lstlisting}
When we start a single producer, we'll see that 10 items are produced and then the producer blocks. When we start a single consumer, we'll immediately see that the producer will continue since the total capacity for the queue has doubled to 20 (2 JVM's times 10 items per queue). 

But since the producer produces 5 times as fast as the consume, the queue will reach its maximum capacity again quickly and it will block. We can can increase the capacity of the cluster by starting new consumers (the processing and the storage capacity increase) or just empty members (the storage capacity increases).

Since Hazelcast 1.9.3, distributed queues are backed by distributed maps. So all the configuration options available to a map, for example storage, are indirectly available for the queue. The value in the map will be the value placed on the queue, and the key of the map (of type Long) will be a global unique id. By default the name to backing map will be the name of the queue prefixed with 'q:', example:
\begin{lstlisting}
    <queue name="queue">
        ...
    </queue>
    <map name="q:queue">
       ...
    </map>
\end{lstlisting}

This naming convention can be overridden by setting the backingMapRef on the queue explicitly, e.g:
\begin{lstlisting}
    <queue name="queue">
       <backingMapRef>somemap</backingMapRef>
       ... 
    </queue>
    <map name="somemap">
        ...
    </map>
\end{lstlisting}
See the Distributed Map chapters for the options available.
TODO: The map can't be retrieved by that name; so either a bug or I'm doing something wrong.
TODO: Is there support to overflow to disk?

\emph{max-size-per-jvm can be violated}. Imagine that there is a queue that is distributed over 2 members and each member contains the maximum number of 1 million queue items. If one member fails, the other member will gain all the queue items stored on the failing member, which leads to a total of 2 million. This is needed to provide failover so that no messages are lost. 

\section{Distributed List}
After the distributed queue, we have the list. As you most likely know, a list is a data-structure where the ordering and occurrence of the elements matters. The distributed List in Hazelcast probably is not a structure you will on a day to day basis, but when you need it, is fine to have. The Hazelcast com.hazelcast.core.IList implements the java.util.List.

We'll demonstrate the list by writing a collection of items in a list on one member, and on another member we are going to print all the elements from that list:
\begin{lstlisting}[language=java]
import com.hazelcast.core.*;
import java.util.*;
public class WriteMember {
    public static void main(String[] args) {
        HazelcastInstance hzInstance = Hazelcast.newHazelcastInstance(null);
        List<String> list = hzInstance.getList("list");
        list.add("Tokyo");
        list.add("Paris");
        list.add("London");
        list.add("New York");
        System.out.println("Putting finished!");
    }
}

import com.hazelcast.core.*;
import java.util.List;
public class ReadMember {
    public static void main(String[] args) {
        HazelcastInstance hzInstance = Hazelcast.newHazelcastInstance(null);
        List<String> list = hzInstance.getList("list");
        for (String s : list) 
            System.out.println(s);
        System.out.println("Reading finished!");
    }
}
\end{lstlisting}
If you first start the write member and after that has completed, you start the read member; the read member will output the following:
\begin{lstlisting}
Tokyo
Paris
London
New York
Reading finished!
\end{lstlisting}
As you can see, the data written to the list by the write member is visible in the read member and you also can see that the order is maintained.

The distributed list also is backed up by the distributed queue and the queue is backed up by a map. The key will be of type Long and the value will by them item placed in the list. To find the map behind the list, prefix the list name with 'c:q:l:'. But no guarantees are given that the map can be accessed this way way in the future, so please beware. The list has no other configuration options.

\section{Distributed Set}
A Set is a collection where every element only occurs ones and where the order of the element doesn't matter. The Hazelcast com.hazelcast.core.ISet implements the java.util.Set.

We'll demonstrate the set by writing a collection of items in a set on one member, and on another member we are going to print all the elements from that set:
\begin{lstlisting}[language=java]
import com.hazelcast.core.*;
import java.util.Set;
public class WriteMember {
    public static void main(String[] args) {
        HazelcastInstance hzInstance = Hazelcast.newHazelcastInstance(null);
        Set<String> set = hzInstance.getSet("set");
        set.add("Tokyo");
        set.add("Paris");
        set.add("London");
        set.add("New York");
        System.out.println("Putting finished");
    }
}

import com.hazelcast.
import java.util.Set;
public class ReadMember {
    public static void main(String[] args) {
         HazelcastInstance hzInstance = Hazelcast.newHazelcastInstance(null);
         Set<String> set = hzInstance.getSet("set");
         for(String s: set)
             System.out.println(s);
         System.out.println("Reading finished!");
     }
}

\end{lstlisting}
If you first start the write member and after that has completed, you start the read member; the read member will output the following:
\begin{lstlisting}
Paris
Tokyo
London
New York
Reading finished!	
\end{lstlisting}
As you can see, the data written to the set by the write member is visible in the read member. As you also can see the order is not maintained since order is not defined by the set.

TODO: Reference to equal/hash of Map.

In Hazelcast the distributed set is implemented based on the distributed map functionality. Unfortunately it isn't very easy to retrieve the backing map. So if you want a set that has the features of the Map it probably is best create a set yourself that is backed by the Hazelcast map. You also can't use the keySet from the distributed map because this will return a new non distributed collection. 

\section{Collection Item Listeners}
The Hazelcast distributed List, Set or Queue implement the com.hazelcast.core.ICollection interface. The nice thing is that Hazelcast enriches the existing collections api with the ability to listen to changes in the collections using the com.hazelcast.core.ItemListener. The ItemListener receives the com.hazelcast.core.ItemListener which not only (potentially) contains the item, but also the member where the changed happened and the type of event (add or remove).

The following example shows an ItemListener that listens to all changes made in a queue, but to listen to changes in a Set or List are similar since all the collections implement the ICollection interface.
\begin{lstlisting}[language=java]
import com.hazelcast.core.*;
public class ItemListenerMember {
    public static void main(String[] args) throws Exception {
        HazelcastInstance hzInstance = Hazelcast.newHazelcastInstance(null);
        ICollection<String> queue = hzInstance.getQueue("queue");
        queue.addItemListener(new ItemListenerImpl<String>(), true);
        System.out.println("ItemListener started");
    }
    private static class ItemListenerImpl<E> implements ItemListener<E> {
        public void itemAdded(ItemEvent<E> itemEvent) {
            System.out.println("Item added:" + itemEvent.getItem());
        }
        public void itemRemoved(ItemEvent<E> itemEvent) {
            System.out.println("Item removed:" + itemEvent.getItem());
        }
    }
}
\end{lstlisting}
We registered the ItemListenerImpl with the addItemListener method using the value 'true'. This is done to make sure that our ItemListenerImpl will get the value that has been added/removed. The reason why this configuration option is available, is that in some cases you only want to be notified that a change happened, but you're not interested in the actual change.

To see that the ItemListener really is working, we'll create a member that makes a change in the queue:
\begin{lstlisting}[language=java]
import com.hazelcast.core.*;
import java.util.concurrent.BlockingQueue;
public class CollectionChangeMember{
    public static void main(String[] args) throws Exception {
        HazelcastInstance hzInstance = Hazelcast.newHazelcastInstance(null);
        BlockingQueue<String> queue = hzInstance.getQueue("queue");
        queue.put("foo");
        queue.put("bar");
        queue.take();
        queue.take();
    }
}
\end{lstlisting}
First start up the itemlistener-member and wait till it displays "ItemListener started". After that start the collection-change-member and you will see the following output in the itemlistener-member:
\begin{lstlisting}
item added:foo
item added:bar
item removed:foo
item removed:bar
\end{lstlisting}
ItemListeners are useful if you need to react upon changes in collections. But realize that listeners are executed asynchronously, so it could be that at the time your listener runs, that the collection has changed again. 

\emph{Ordering} All events are ordered, meaning, listeners will receive and process the events in the order they are actually occurred. TODO: Is the ordering only guaranteed within the member, or is there a cluster wide ordering?

\section{Gotcha's}
\emph{Iterator stability}: iterators on collections are weakly consistent; meaning that when a collection changes while creating the iterator, you could encounter duplicates or miss element. Changes on that iterator will not result in changes on the collection. An iterator doesn't need to reflect the actual state and will not throw a ConcurrentModifcationException.

\section{What is next?}
The api shown in these examples is only a subsection of what Hazelcast provides.