\chapter{Network configuration}

Hazelcast can run perfectly within a single JVM and this is excellent for development and to speed up testing. But Hazelcast true strength becomes apparent when a cluster of JVM's running on multiple machines is created. Having a cluster of machines makes Hazelcast resilient to failure; if one machine fails, the data will failover to another machine as if nothing happened. It also makes Hazelcast scalable; just add extra machines to the cluster, to gain additional capacity. Creating clusters can be done by configuring the network settings.

To configure the networking settings, we are going to make use of the following minimalistic Hazelcast node.
\begin{lstlisting}[language=java]
import com.hazelcast.core.Hazelcast;
public class Node {
    public static void main(String[] args) {
        Hazelcast.newHazelcastInstance(null);
    }
}
\end{lstlisting}

In this chapter we are going to rely on the hazelcast.xml config file to set the networking options, but it can also be done programmatically. Within the config file there is a network section that can be configured:
\begin{lstlisting}[language=xml]
<hazelcast>
    ...
    <network>
        ...  
   </network>
    ...
</hazelcast>
\end{lstlisting}

\section{Port}
One of the most basic configuration settings is the port Hazelcast is going to use for communication between the nodes. This can be done by the 'port' property in the network configuration which defaults to 5701. If the port is already in use and if auto-increment property is true, it is the default, Hazelcast will automatically try to find next port. Example:
\begin{lstlisting}[language=xml]
<hazelcast>
    <network>
        <port auto-increment="true">5701</port>
   </network>
</hazelcast>
\end{lstlisting}
In this example you can see the explicit set values, but you don't need to specify them since they are also the default. If you start the node, you will get output like:

\begin{lstlisting}
...
Jan 19, 2013 10:58:46 AM com.hazelcast.impl.AddressPicker
INFO: Prefer IPv4 stack is true.
Jan 19, 2013 10:58:46 AM com.hazelcast.impl.AddressPicker
INFO: Picked Address[192.168.1.104]:5701, using socket ServerSocket[addr=/0.0.0.0,localport=5701], bind any local is true
Jan 19, 2013 10:58:47 AM com.hazelcast.system
INFO: [192.168.1.104]:5701 [dev] Hazelcast Community Edition 2.5 (20130116) starting at Address[192.168.1.104]:5701
Jan 19, 2013 10:58:47 AM com.hazelcast.system
INFO: [192.168.1.104]:5701 [dev] Copyright (C) 2008-2012 Hazelcast.com
Jan 19, 2013 10:58:47 AM com.hazelcast.impl.LifecycleServiceImpl
INFO: [192.168.1.104]:5701 [dev] Address[192.168.1.104]:5701 is STARTING
Jan 19, 2013 10:58:47 AM com.hazelcast.impl.Node
WARNING: [192.168.1.104]:5701 [dev] No join method is enabled! Starting standalone.
...
\end{lstlisting}
As you can see, the port is 5701. If another Node is started, you will get output like this:
\begin{lstlisting}
...
INFO: Prefer IPv4 stack is true.
Jan 19, 2013 11:01:09 AM com.hazelcast.impl.AddressPicker
INFO: Picked Address[192.168.1.104]:5702, using socket ServerSocket[addr=/0.0.0.0,localport=5702], bind any local is true
Jan 19, 2013 11:01:09 AM com.hazelcast.system
INFO: [192.168.1.104]:5702 [dev] Hazelcast Community Edition 2.5 (20130116) starting at Address[192.168.1.104]:5702
Jan 19, 2013 11:01:09 AM com.hazelcast.system
INFO: [192.168.1.104]:5702 [dev] Copyright (C) 2008-2012 Hazelcast.com
Jan 19, 2013 11:01:09 AM com.hazelcast.impl.LifecycleServiceImpl
INFO: [192.168.1.104]:5702 [dev] Address[192.168.1.104]:5702 is STARTING
Jan 19, 2013 11:01:09 AM com.hazelcast.impl.Node
WARNING: [192.168.1.104]:5702 [dev] No join method is enabled! Starting standalone.
Jan 19, 2013 11:01:09 AM com.hazelcast.impl.Node
WARNING: [192.168.1.104]:5702 [dev] Config seed port is 5701 and cluster size is 1. Some of the ports seem occupied!
Jan 19, 2013 11:01:09 AM com.hazelcast.impl.LifecycleServiceImpl
INFO: [192.168.1.104]:5702 [dev] Address[192.168.1.104]:5702 is STARTED...
\end{lstlisting}
As you can see the port of the second node is 5702. If you look carefully, at the end you see a warning about cluster size is 1. This is because the created nodes are currently not part of a cluster, but are standalone nodes. To create a cluster, see the Node Discovery section.

\section{Node discovery}
Hazelcast currently supports 3 mechanisms for Hazelcast nodes to discover each other and join the cluster. Without this discovery mechanisms, the node's will not form a cluster, but will remain stand alone Hazelcast instances. The discovery mechanisms are:
\begin{enumerate}
\item Multicast:
\item TCP/IP Cluster:
\item Amazon EC2:
\end{enumerate}

\subsection{Configuring Hazelcast for Multicast}
With multicast discovery a node will send a message to all node's that listen on the multi-cast port. It is the easiest configuration for testing. But in a lot of production environments, multicast is not supported or desirable.

A very minimalistic multicast configuration is the following:
\begin{lstlisting}[language=xml]
<hazelcast>
    <network>
        <join><multicast enabled="true"/></join>
    </network>
</hazelcast>
\end{lstlisting}

If you start one node, you will see output like this:
\begin{lstlisting}
Sep 5, 2012 6:39:31 PM com.hazelcast.impl.AddressPicker
INFO: Prefer IPv4 stack is true.
Sep 5, 2012 6:39:31 PM com.hazelcast.impl.AddressPicker
INFO: Picked Address[10.37.129.2]:5701, using socket ServerSocket[addr=/0.0.0.0,localport=5701], bind any local is true
Sep 5, 2012 6:39:31 PM com.hazelcast.system
INFO: [10.37.129.2]:5701 [dev] Hazelcast Community Edition 2.3 (20120828) starting at Address[10.37.129.2]:5701
Sep 5, 2012 6:39:31 PM com.hazelcast.system
INFO: [10.37.129.2]:5701 [dev] Copyright (C) 2008-2012 Hazelcast.com
Sep 5, 2012 6:39:31 PM com.hazelcast.impl.LifecycleServiceImpl
INFO: [10.37.129.2]:5701 [dev] Address[10.37.129.2]:5701 is STARTING
Sep 5, 2012 6:39:33 PM com.hazelcast.impl.MulticastJoiner
INFO: [10.37.129.2]:5701 [dev] 

Members [1] {
    Member [10.37.129.2]:5701 this
}

Sep 5, 2012 6:39:34 PM com.hazelcast.impl.LifecycleServiceImpl
INFO: [10.37.129.2]:5701 [dev] Address[10.37.129.2]:5701 is STARTED
\end{lstlisting}	

As you can see the node is started and currently the cluster only has a single member. If you start another node on the same machine, on the console of the first node the following will be added to the output:
\begin{lstlisting}
Sep 5, 2012 6:39:38 PM com.hazelcast.nio.SocketAcceptor
INFO: [10.37.129.2]:5701 [dev] 5701 is accepting socket connection from /10.37.129.2:64916
Sep 5, 2012 6:39:38 PM com.hazelcast.nio.ConnectionManager
INFO: [10.37.129.2]:5701 [dev] 5701 accepted socket connection from /10.37.129.2:64916
Sep 5, 2012 6:39:44 PM com.hazelcast.cluster.ClusterManager
INFO: [10.37.129.2]:5701 [dev] 

Members [2] {
    Member [10.37.129.2]:5701 this
    Member [10.37.129.2]:5702
}
\end{lstlisting}	
As you can see, the first node can see the second node. And if we look at the end of logging for the second node, we'll find something similar:
\begin{lstlisting}
Members [2] {
    Member [10.37.129.2]:5701
    Member [10.37.129.2]:5702 this
}

Sep 5, 2012 6:39:45 PM com.hazelcast.impl.LifecycleServiceImpl
INFO: [10.37.129.2]:5702 [dev] Address[10.37.129.2]:5702 is STARTED
\end{lstlisting}		

After this point, we now have a 2 node Hazelcast cluster running on a single machine. Try creating another few nodes. And you can of course also try to connect multiple nodes over a network;

If you don't get the see multiple members joining each other, than it is likely because multicast has been disabled. If you are on a *NIX environment, you can try to execute the command 'ifconfig' and look for 'MULTICAST'. 

\begin{lstlisting}[language=xml]
<hazelcast>
    <network>
        <join>
            <multicast enabled="true">
                <multicast-group>224.2.2.3</multicast-group>
                <multicast-port>54327</multicast-port>
                <multicast-time-to-live-seconds>32</multicast-time-to-live-seconds>
                <multicast-timeout-seconds></multicast-timeout-seconds>
                <trusted-interfaces>???</trusted-interfaces>
            </multicast>
        </join>
    </network>
</hazelcast>
\end{lstlisting}[language=xml]

The Hazelcast multicast functionality is build on the java.net.MulticastSocket.

TODO: So by setting the multicast-group or the multicast-port, you can have separate Hazelcast clusters within the same network. Which is the preferred way?

multicast-group:
With multicast a process is part of the multicast group. Only other processes that are part of the same group will receive each others messages. The multicast group ip address doesn't conflict with normal unicast ip addresses since they have a specific range that is excluded from normal unicast usage. 224.0.0.0 to 239.255.255.255 (inclusive)
defaults:224.2.2.3. The address 224.0.0.0 is reserved and should not be used.

multicast-port: The port of the multicast socket where the Hazelcast Node listens and sends discovery messages on. Unlike normal unicast sockets where only a single process can listen to a port, with multicast sockets multiple processes can listen to the same port. So you don't need to be worried about having multiple Hazelcast nodes that run on the same JVM are going to conflict.
defaults:54327

TODO: Why are we talking about time to live in seconds? TTL on multicast isn't interpreted in seconds.
multicast-time-to-live-seconds:
defaults:32
/**
   * Set the default time-to-live for multicast packets sent out
   * on this <code>MulticastSocket</code> in order to control the 
   * scope of the multicasts.
   *
   * <P> The ttl <B>must</B> be in the range <code> 0 <= ttl <=
   * 255</code>


multicast-timeout-seconds:
defaults:2

trusted-interfaces:
defaults:


\subsection{Configuring Hazelcast for full TCP/IP cluster}

Clouds often don't provide multicast.

The idea is that there should be one or more well known machines.

It can be configured through setting the hostname or can be configured through the interface. 

If multicast is not preferred way of discovery for your environment, then you can configure Hazelcast for full TCP/IP cluster. As configuration below shows, while enable attribute of multicast is set to false, tcp-ip has to be set to true. For the none-multicast option, all or subset of cluster members' hostnames and/or ip addresses must be listed. Note that all of the cluster members don't have to be listed there but at least one of them has to be active in cluster when a new member joins. The tcp-ip tag accepts an attribute called "conn-timeout-seconds". The default value is 5. Increasing this value is recommended if you have many IP's listed and members can not properly build up the cluster.

\begin{lstlisting}[language=xml]
<hazelcast>
    ...
    <network>
        <port auto-increment="true">5701</port>
        <join>
            //why are the multicast-group and port still set?
            <multicast enabled="false"/>
            <tcp-ip enabled="true">
                <hostname>machine1</hostname>
                <hostname>machine2</hostname>
                <hostname>machine3:5799</hostname>
                <interface>192.168.1.0-7</interface>     
                <interface>192.168.1.21</interface> 
            </tcp-ip>
        </join>
        ...
    </network>
    ...
</hazelcast>
\end{lstlisting}

<required-member></required-member>: defaults to 127.0.0.1


\subsection{Configuring Hazelcast for EC2 Auto Discovery}

Hazelcast supports EC2 Auto Discovery as of 1.9.4. It is useful when you don't want or can't provide the list of possible IP addresses. Here is a sample configuration: Disable join over multicast and tcp/ip and enable aws. Also provide the credentials. The aws tag accepts an attribute called "conn-timeout-seconds". The default value is 5. Increasing this value is recommended if you have many IP's listed and members can not properly build up the cluster.

\begin{lstlisting}[language=xml]
<hazelcast>
    ...
    <network>
       <join>
           <multicast enabled="false"/>
           <tcp-ip enabled="false">
              todo: why is interface set?
              <interface>192.168.1.2</interface>
            </tcp-ip>
    
            <aws enabled="true">
               <access-key>my-access-key</access-key>
               <secret-key>my-secret-key</secret-key>
               <region>us-west-1</region>                              <!-- optional, default is us-east-1 -->
               <security-group-name>hazelcast-sg</security-group-name> <!-- optional -->
               <tag-key>type</tag-key>                                  <!-- optional -->
               <tag-value>hz-nodes</tag-value>                          <!-- optional -->
            </aws>
       </join>
    </network>
</hazelcast>
\end{lstlisting}

\section{Creating Separate Clusters}

By specifying group-name and group-password, you can separate your clusters in a simple way; dev group, production group, test group, app-a group etc... 

todo: password is optional, it defaults to 'drowssap'

\begin{verbatim}
<hazelcast>
    <group>
        <name>dev</name>
        <password>dev-pass</password>
    </group>
    ...
</hazelcast>
\end{verbatim}
You can also set the groupName with Config API. JVM can host multiple Hazelcast instances (nodes). Each node can only participate in one group and it only joins to its own group, does not mess with others. Following code creates 3 separate Hazelcast nodes, h1 belongs to app1 cluster, while h2 and h3 are belong to app2 cluster.

\begin{lstlisting}
Config configApp1 = new Config();
configApp1.getGroupConfig().setName("app1");

Config configApp2 = new Config();
configApp2.getGroupConfig().setName("app2");

HazelcastInstance h1 = Hazelcast.newHazelcastInstance(configApp1);
HazelcastInstance h2 = Hazelcast.newHazelcastInstance(configApp2);
HazelcastInstance h3 = Hazelcast.newHazelcastInstance(configApp2);
\end{lstlisting}

\section{Specifying network interfaces}

You can also specify which network interfaces that Hazelcast should use. Servers mostly have more than one network interface so you may want to list the valid IPs. Range characters ('*' and '-') can be used for simplicity. So 10.3.10.*, for instance, refers to IPs between 10.3.10.0 and 10.3.10.255. Interface 10.3.10.4-18 refers to IPs between 10.3.10.4 and 10.3.10.18 (4 and 18 included). If network interface configuration is enabled (disabled by default) and if Hazelcast cannot find an matching interface, then it will print a message on console and won't start on that node.

\begin{lstlisting}[language=xml]
<hazelcast>
    ...
    <network>
        ....
        <interfaces enabled="true">
            <interface>10.3.16.*</interface> 
            <interface>10.3.10.4-18</interface> 
            <interface>192.168.1.3</interface>         
        </interfaces>    
    </network>
    ...
</hazelcast>
\end{lstlisting}

\section{Partition Group Configuration}
Hazelcast distributes key objects into partitions (blocks) using a consistent hashing algorithm and those partitions are assigned to nodes. That means an entry is stored in a node which is owner of partition to that entry's key is assigned. Number of total partitions is default 271 and can be changed with configuration property hazelcast.map.partition.count. Along with those partitions, there are also copies of them as backups. Backup partitions can have multiple copies due to backup count defined in configuration, such as first backup partition, second backup partition etc. As a rule, a node can not hold more than one copy of a partition (ownership or backup). By default Hazelcast distributes partitions and their backup copies randomly and equally among cluster nodes assuming all nodes in the cluster are identical.

What if some nodes share same JVM or physical machine or chassis and you want backups of these nodes to be assigned to nodes in another machine or chassis? What if processing or memory capacities of some nodes are different and you do not want equal number of partitions to be assigned to all nodes?

You can group nodes in same JVM (or physical machine) or nodes located in the same chassis. Or you can group nodes to create identical capacity. We call these groups as partition groups. This way partitions are assigned to those partition groups instead of single nodes. And backups of these partitions are located in another partition group.

When you enable partition grouping, Hazelcast presents two choices to configure partition groups at the moments.

First one is to group nodes automatically using IP addresses of nodes, so nodes sharing same network interface will be grouped together.

\begin{lstlisting}
<partition-group enabled="true" group-type="HOST_AWARE" />
Config config = ...;
PartitionGroupConfig partitionGroupConfig = config.getPartitionGroupConfig();
partitionGroupConfig.setEnabled(true).setGroupType(MemberGroupType.HOST_AWARE);
\end{lstlisting}

Second one is custom grouping using Hazelcast's interface matching configuration. This way, you can add different and multiple interfaces to a group. You can also use wildcards in interface addresses.

\begin{lstlisting}
<partition-group enabled="true" group-type="CUSTOM">
    <member-group>
        <interface>10.10.0.*</interface>
        <interface>10.10.3.*</interface>
        <interface>10.10.5.*</interface>
    </member-group>
    <member-group>
        <interface>10.10.10.10-100</interface>
        <interface>10.10.1.*</interface>
        <interface>10.10.2.*</interface>
    </member-group
</partition-group>
Config config = ...;
PartitionGroupConfig partitionGroupConfig = config.getPartitionGroupConfig();
partitionGroupConfig.setEnabled(true).setGroupType(MemberGroupType.CUSTOM);

MemberGroupConfig memberGroupConfig = new MemberGroupConfig();
memberGroupConfig.addInterface("10.10.0.*")
    .addInterface("10.10.3.*").addInterface("10.10.5.*");

MemberGroupConfig memberGroupConfig2 = new MemberGroupConfig();
memberGroupConfig2.addInterface("10.10.10.10-100")
    .addInterface("10.10.1.*").addInterface("10.10.2.*");

partitionGroupConfig.addMemberGroupConfig(memberGroupConfig);
partitionGroupConfig.addMemberGroupConfig(memberGroupConfig2);
\end{lstlisting}

\section{IPv6 Support}

\section{SSL}
TODO: What is the difference between SSL and encryption

In a production environment often you want to prevent that the communication between Hazelcast nodes can be tampered or can be read, with because it could contains sensitive information. Luckily Hazelcast provides a solution for that by enabling SSL encryption.

The basic functionality is provided by 'com.hazelcast.nio.ssl.SSLContextFactory' interface and configurable through the the SSL section in network configuration. Luckily Hazelcast provides a default implementation called the 'com.hazelcast.nio.ssl.BasicSSLContextFactory' which we are going to use for the example:
\begin{lstlisting}[language=xml]
<hazelcast>
    <network>
        <join><multicast enabled="true"/></join>
        <ssl enabled="true">
            <factory-class-name>com.hazelcast.nio.ssl.BasicSSLContextFactory</factory-class-name>
            <properties>
                <property name="keyStore">keyStore.jks</property>
                <property name="keyStorePassword">password</property>
            </properties>
        </ssl>
    </network>
</hazelcast>
\end{lstlisting}
The 'keyStore' is the path to the keyStore. [todo: is it possible to provide a classpath reference so it can be included in the jar, and is this a desirable practice?] and the 'keyStorePassword' is the password of the keystore. In the example code you can find an already created keystore and also the documentation to create one yourself.

When you start a node, you will see that SSL is enabled:
\begin{lstlisting}
Jan 19, 2013 4:00:43 PM com.hazelcast.nio.ConnectionManager
INFO: [192.168.1.104]:5701 [dev] SSL is enabled
\end{lstlisting}

There are some additional properties that can be set on the BasicSSLContextFactory:
\begin{enumerate}
\item keyManagerAlgorithm: defaults to 'SunX509'.
\item trustManagerAlgorithm: defaults to 'SunX509'.
\item protocol: defaults to 'TLS'
\end{enumerate}
Another way to configure the keyStore/keyStorePassword is through the 'javax.net.ssl.keyStore' and 'javax.net.ssl.keyStorePassword' system properties.

\section{Encryption}
[todo: why is there encryption and the ssl option? So when to select one or the other?]
[todo: are there any performance implications?]
[todo: when to use async or sync.]
[todo: encryption and SSL can't be combined with each other. ]
[todo: asynchronous and synchronous can't be combined with each other.]

\begin{enumerate}
\item asymmetric encryption
\item symmetric encryption
\end{enumerate}

\subsection{Asymmetric Encryption}
Asymmetric encryption relies on a public/private-key pair. Data is encrypted by the sender using the public, a key everybody is allowed to have, and decrypted using the private key of the receiver. 

\begin{lstlisting}[language=xml]

\end{lstlisting}

[todo:the following algorithms are available RSA/NONE/PKCS1PADDING]
[todo:the none algorithm disables the encryption?]
[todo:how to keep the keystores in sync? Since each keystore needs to be updated when a member joins]

\subsection{Symmetric Encryption}
[todo:which algorithms are available]

\begin{lstlisting}[language=xml]
<hazelcast>
    <network>
        <join>
            <multicast enabled="true"/>
        </join>
        <symmetric-encryption enabled="true">
            <algorithm>PBEWithMD5AndDES</algorithm>
            <salt>somesalt</salt>
            <password>somepass</password>
            <iteration-count>19</iteration-count>
        </symmetric-encryption>
    </network>
</hazelcast>
\end{lstlisting}
When we start 2 Nodes using the configuration, we'll see that the symmetric encryption is activated:
\begin{lstlisting}
Jan 20, 2013 9:22:08 AM com.hazelcast.nio.SocketPacketWriter
INFO: [192.168.1.104]:5702 [dev] Writer started with SymmetricEncryption
Jan 20, 2013 9:22:08 AM com.hazelcast.nio.SocketPacketReader
INFO: [192.168.1.104]:5702 [dev] Reader started with SymmetricEncryption
\end{lstlisting}

\section{Socket Interceptor}

\section{Network Partitioning (Split-Brain Syndrome)}

\section{Firewall}
When a Hazelcast member connects to another Hazelcast member, it binds to server port 5701 (see the port configuration section). This port receives the inbound traffic. On the client side also a port needs to be opened for the outbound traffic. By default this will be an 'ephemeral' port since we don't care about which port is being used as long as it is free. The problem is that this kind of behavior can lead to security issues since there is no control on the ports being used and therefor the firewall needs to expose all ports for outbound traffic. 

Luckily Hazelcast is able to restrict the outbound ports by configuring them. For example if we want to allow the port range 30000-31000 for outbound traffic, it can be configured like this:
\begin{lstlisting}[language=xml]
<hazelcast>
    <network>
        <join><multicast enabled="true"/></join>
        <outbound-ports>
            <ports>30000-31000</ports>
        </outbound-ports>
    </network>
</hazelcast
\end{lstlisting}
To demonstrate the outbound ports configuration, start 2 Hazelcast node's with the configuration and when the nodes are fully started, execute 'sudo lsof -i | grep java'. Below you can see the cleaned output of the command:
\begin{lstlisting}
java 46117 IPv4 TCP *:5701 (LISTEN)
java 46117 IPv4 TCP 172.16.78.1:5701->172.16.78.1:30609 (ESTABLISHED)
java 46120 IPv4 TCP *:5702 (LISTEN)
java 46120 IPv4 TCP 172.16.78.1:30609->172.16.78.1:5701 (ESTABLISHED)
\end{lstlisting}
As you can see there are 2 java processes: 46117 and 46120 that listen on port 5701 and 5702 (inbound traffic). You can also see that java process 46120 is using port 30609 for outbound traffic.

Apart from specifying port ranges, you can also specify single ports. And multiple port configurations can be combined either by separating them by comma or by providing multiple 'ports' sections. So if you want to use port 30000,30005 and portrange 31000 till 32000, you could say the following: <ports>30000,30005,31000-32000</ports>. 

\subsection{iptables}
If you are making use of iptables the following rule can be added to allow for outbound traffic from ports 33000-31000:
\begin{lstlisting}
iptables -A OUTPUT -p TCP --dport 33000:31000 -m state --state NEW -j ACCEPT
\end{lstlisting}
and the following rule allows for incoming traffic from any address to port 5701.
\begin{lstlisting}
iptables -A INPUT -p tcp -d 0/0 -s 0/0 --dport 5701 -j ACCEPT
\end{lstlisting}

\section{What is next}
Lorem ipsum dolor sit amet, consectetur adipiscing elit. 

[todo: comment about generating the hazelcast.xml to deal with limitations in the join mechanism. You could generate
a hazelcast.xml based on database content containing the members.]
