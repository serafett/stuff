\chapter{Hazelcast Clients}

[todo: near caching]
[todo: tell something about what happens when cluster not yet available.. blocking? timeout? exception?]

Till so far the examples showed members that were full participants in the cluster; so they will known about others and they will host partitions. But in some cases you only want to connect to the cluster to read/write data from the cluster or execute operations, but you don't want to participate as a full member in the cluster; in other words you want to have a client.

We are going to implement a client example where a message will be put on a queue by a client and taken from the queue by a full member:
\begin{lstlisting}[language=java]
public class FullMember {
   public static void main(String[] args)throws Exception{
      HazelcastInstance hz = Hazelcast.newHazelcastInstance();
      BlockingQueue<String> queue = hz.getQueue("queue");
      for(;;) 
         System.out.println(queue.take());
   }
}
\end{lstlisting}

With the client one can connect to the cluster purely as a client and not have any of the responsibilities a normal cluster member has. When a Hazelcast operation is performed by a client, it is forwarded to a cluster member where it will be processed. A client only needs to have the hazelcast-client.jar on the classpath (+/- 150 kB depending on the version) [todo: verify], the normal Hazelcast jar is not needed. Underneath you can see a client example:
\begin{lstlisting}[language=java]
public class Client {
   public static void main(String[] args) throws Exception {
      ClientConfig cfg = new ClientConfig()
         .addAddress("127.0.0.1");
      HazelcastInstance client = HazelcastClient.newHazelcastClient(cfg);
      BlockingQueue<String> queue = client.getQueue("queue");
      queue.put("Hello");
      System.out.println("Message send by Client!");
   }
}
\end{lstlisting}
The client HazelcastInstance is created based on the com.hazelcast.client.ClientConfig. This config is configured with 127.0.0.1 as address since the full member will be running on the same machine as the client.

First start the full member and then start the client, and we can see that "Hello" is printed by the full member. We'll also see that the client never appears as a member in the member listing, printed the FullMember.

\section{Configuration Options}
In the client example, we did a minimal configuration of the ClientConfig and relied on defaults, but there is a lot that can be configured:
\begin{enumerate}
\item addresses: the known addresses of the cluster. It doesn't need to include all addresses, only enough to make sure that some will be online. see [failover]
\item credentials: can be used to do authentication and authorization. This functionality is only available in the Enterprise version of Hazelcast.
\item group config: see [group config section]
\item connection timeout: the number of ms to wait before giving up while connecting. Defaults to 300000 ms (5 minutes).
\item initial connection attempt limit:the  number of connection attempts the client makes the initial connection to the cluster. Defaults to 1.
\item reconnect attempt limit: the number of reconnection attempts the client makes when a connection fails. Defaults to 1.
\item reconnect timeout: the number of ms to wait before giving up while reconnecting. Defaults to 5000 ms.
\item listeners: enabled listening to cluster state using a LifecycleListener, DistributedObjectListener or MembershipListener.
\item loadBalancer: see LoadBalancing for more information. Defaults to RoundRobinLB.
\item smart: [todo] defaults to true
\item retryOperation: [todo] default to true.
\item poolSize: [todo] defaults
\item meapNearCacheConfigs: [todo] defaults to empty.
\end{enumerate}

\section{LoadBalancing}.
When a client connects to the cluster, it will have access to the full list of members and it will be kept in sync, even if the ClientConfig only has a subset of members. If a operation needs to be sent to a specific member, it will directly be sent to that member. If a operation can be executed on any member, Hazelcast does automatic load balancing over all members in the cluster. 

One of the very new cool features of Hazelcast 3.0 is that the routing mechanism is pulled out into an interface:
\begin{lstlisting}[language=java]
public interface LoadBalancer {
   void init(HazelcastInstance hz);
   Member next();
}
\end{lstlisting}
This means that if you have specific routing requirements, e.g. load balance on cpu load, memory load, queue sizes etc, these requirements can be met by creating a custom LoadBalancer implementation. I'm sure that in the next releases of Hazelcast some of these implementations will be provided out of the box. If you are going to implement a custom loadBalancer, you can listen to member changes using:
\begin{lstlisting}[language=java]
Cluster cluster = hz.getCluster();
cluster.addMembershipListener(thelistener);
\end{lstlisting}

The MembershipListener functionality makes it easy to to create a deterministic loadbalancer since the MembershipListener:
\begin{enumerate}
\item will not be called concurrently
\item init method will be called with a set of all current members
\item no events will be lost between calling init and memberAdded or memberRemoved
\item memberAdded and memberRemoved method will be called in the order the events happened within the cluster. There is a total ordering of membership events since they will be coordinated from the master node.
\end{enumerate}

LoadBalancer instances should not be shared between clients; every client should gets its own instance and the loadbalancer can be configured from the ClientConfig. The clients are thread-safe, and therefore can be called concurrently, so you don't need to pool them.

\section{Failover} 
In a production environment you want the client to support failover to increase high availability. This is realized by in 2 parts. The first part is configuring multiple member addresses in the ClientConfig. As long as one of these members is online, the client will be able to connect to the cluster. The second part is the responsibility of the LoadBalancer implementation. It can register itself as a MembershipListener and receives a list of all members in the cluster and will be notified if members are added or are removed. The LoadBalancer can use this update to date list of member addresses for routing.

\section{Group Configuration}
To prevent clients from joining a cluster, it is possible to configure the cluster group the client is able to connect to. On this cluster group the group name and the password can be set:
\begin{lstlisting}[language=java]
ClientConfig config = new ClientConfig()
    .addAddress("127.0.0.1");
config.getGroupConfig()
    .setName("group1")
    .setPassword("thepassword");
HazelcastInstance client = HazelcastClient.newHazelcastClient(config);
\end{lstlisting}
The groupname defaults to 'dev' and the password defaults to 'dev-pass'. For more information see [network configuration: cluster groups].

\section{Sharing classes}
In some cases you need to share classes between the client and the server. Of course you can give all the classes from the server to the client, but in some cases this is undesirable; perhaps it would increase tight coupling, security/copyrights issues, increased client size etc. If you don't want to share all the classes of the server with the client, create a separate project (in Maven terms this could be a module) and share this project between client and server. I normally call this project the api project since it clearly defines the API of the system. 

One advice; watch out with sharing domain objects between client and server. This can cause a tight coupling since the client starts to see the internals of your domain objects. A recommended practice is to introduce special objects that are optimized for client/server exchange; Data Transfer Objects (DTO's). They cause some duplication, but having some duplication is better to deal with than tight coupling, which can make a system very fragile.

\section{What happened to the lite member?}
If you have been using Hazelcast 2.x you might remember the option to create either a lite member or a native member. A lite member is seen as part of the cluster, although it doesn't host any partitions. Because it is part of the cluster, it knows about the other members. And therefor it knows about routing requests to correct member, which could improve performance compared to the native client. 

The lite member also has some serious drawbacks:
\begin{enumerate}
\item because lite members are seen as members in the cluster, they send heartbeats/pings to each other and check each others statuses continuously. So if the number of lite members compared to the number of real members is high, and clients join/leave frequently, it can influence the health of the cluster. 
\item although a lite member doesn't host any partitions, it will run tasks from the Hazelcast executors. Personally I always found this undesirable since you don't want to have a client run cluster tasks.
\end{enumerate}
With Hazelcast 3.x a client always knows about all members and therefor can route requests to the correct member, without being part of the cluster. 

\section{Good to know}

\emph{No Encryption:} In Hazelcast 3.0 it isn't possible to encrypt communication between client and cluster. This means that all network traffic, which not only includes normal operations like a map.put but also passwords in credentials and GroupConfig, can be read and potentially modified. If this is an issue, it is best not expose the client but manually expose the Hazelcast operations using a more secure remoting technology. 

\emph{SPI:} The Hazelcast client can also call SPI operations, see [SPI chapter]. But you need to make sure that the client has access the the appropriate classes and interfaces.  [todo: currently it doesn't work.]

\section{What is next}
In this short chapter we explained a few different ways to connect to a Hazelcast cluster using a client. But there are more client solutions available: like the C\# client, C++ client, Memcache Client and the Rest Client. For more information check the Client chapter of the Hazelcast reference manual.