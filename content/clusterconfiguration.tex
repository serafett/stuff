\chapter{Network configuration}

TODO: IPV6

To configure the cluster options, we are going to make use of the following minimalistic Hazelcast node.
\begin{lstlisting}[language=java]
import com.hazelcast.core.Hazelcast;

public class Node {
    public static void main(String[] args) {
        Hazelcast.newHazelcastInstance(null);
    }
}
\end{lstlisting}

Hazelcast network configuration can be done through the hazelcast.xml config file (it also can be done programmatically).
\begin{lstlisting}[language=xml]
<hazelcast>
    ...
    <network>
        ...  
   </network>
    ...
</hazelcast>
\end{lstlisting}


network.port:
this is the port the Hazelcast uses for internal communication. If that port already is taken, e.g because another Hazelcast instance is running on that JVM, Hazelcast will automatically try to find another port. This is done by incrementing the port number
\begin{lstlisting}[language=xml]
<hazelcast>
    ...
    <network>
        <port auto-increment="true">5701</port>
        ...  
   </network>
    ...
</hazelcast>
\end{lstlisting}
It defaults to 5701; so if you are fine with that, you can leave it as it is.

\section{Node discovery}
Hazelcast currently supports 3 mechanisms for Hazelcast nodes to discover each other and join the cluster. Without this discovery mechanism, the node's will not form a cluster together, but will remain 1 node clusters. The discovery mechanisms are:
\begin{enumerate}
\item Multicast:
\item TCP/IP Cluster:
\item Amazon EC2:
\end{enumerate}

\subsection{Configuring Hazelcast for Multicast}

With multicast discovery a node will send a message to all node's that listen on the multi-cast port. It is the easiest configuration for testing. But in a lot of production environments, multicast is not allowed.

A very minimalistic multicast configuration is the following:
\begin{lstlisting}[language=xml]
<hazelcast>
    <network>
        <join>
            <multicast enabled="true"/>
        </join>
    </network>
</hazelcast>
\end{lstlisting}

If you start one node, you will see output like this:
\begin{verbatim}
Sep 5, 2012 6:39:31 PM com.hazelcast.impl.AddressPicker
INFO: Prefer IPv4 stack is true.
Sep 5, 2012 6:39:31 PM com.hazelcast.impl.AddressPicker
INFO: Picked Address[10.37.129.2]:5701, using socket ServerSocket[addr=/0.0.0.0,localport=5701], bind any local is true
Sep 5, 2012 6:39:31 PM com.hazelcast.system
INFO: [10.37.129.2]:5701 [dev] Hazelcast Community Edition 2.3 (20120828) starting at Address[10.37.129.2]:5701
Sep 5, 2012 6:39:31 PM com.hazelcast.system
INFO: [10.37.129.2]:5701 [dev] Copyright (C) 2008-2012 Hazelcast.com
Sep 5, 2012 6:39:31 PM com.hazelcast.impl.LifecycleServiceImpl
INFO: [10.37.129.2]:5701 [dev] Address[10.37.129.2]:5701 is STARTING
Sep 5, 2012 6:39:33 PM com.hazelcast.impl.MulticastJoiner
INFO: [10.37.129.2]:5701 [dev] 

Members [1] {
    Member [10.37.129.2]:5701 this
}

Sep 5, 2012 6:39:34 PM com.hazelcast.impl.LifecycleServiceImpl
INFO: [10.37.129.2]:5701 [dev] Address[10.37.129.2]:5701 is STARTED
\end{verbatim}	

As you can see the node is started and currently the cluster only has a single member. If you start another node on the same machine, on the console of the first node the following will be added to the output:

\begin{verbatim}
Sep 5, 2012 6:39:38 PM com.hazelcast.nio.SocketAcceptor
INFO: [10.37.129.2]:5701 [dev] 5701 is accepting socket connection from /10.37.129.2:64916
Sep 5, 2012 6:39:38 PM com.hazelcast.nio.ConnectionManager
INFO: [10.37.129.2]:5701 [dev] 5701 accepted socket connection from /10.37.129.2:64916
Sep 5, 2012 6:39:44 PM com.hazelcast.cluster.ClusterManager
INFO: [10.37.129.2]:5701 [dev] 

Members [2] {
    Member [10.37.129.2]:5701 this
    Member [10.37.129.2]:5702
}
\end{verbatim}	
As you can see, the first node can see the second node. And if we look at the end of logging for the second node, we'll find something similar.

\begin{verbatim}
Members [2] {
    Member [10.37.129.2]:5701
    Member [10.37.129.2]:5702 this
}

Sep 5, 2012 6:39:45 PM com.hazelcast.impl.LifecycleServiceImpl
INFO: [10.37.129.2]:5702 [dev] Address[10.37.129.2]:5702 is STARTED
\end{verbatim}		

After this point, we now have a 2 node Hazelcast cluster running on a single machine. Try creating another few nodes. And you can of course also try to connect multiple nodes over a network;

If you don't get the see multiple members joining each other, than it is likely because multicast has been disabled. If you are on a *NIX environment, you can try to execute the command 'ifconfig' and look for 'MULTICAST'. 

\begin{lstlisting}[language=xml]
<hazelcast>
    <network>
        <join>
            <multicast enabled="true">
                <multicast-group>224.2.2.3</multicast-group>
                <multicast-port>54327</multicast-port>
                <multicast-time-to-live-seconds>32</multicast-time-to-live-seconds>
                <multicast-timeout-seconds></multicast-timeout-seconds>
                <trusted-interfaces>???</trusted-interfaces>
            </multicast>
        </join>
    </network>
</hazelcast>
\end{lstlisting}[language=xml]

The Hazelcast multicast functionality is build on the java.net.MulticastSocket.

TODO: So by setting the multicast-group or the multicast-port, you can have separate Hazelcast clusters within the same network. Which is the preferred way?

multicast-group:
With multicast a process is part of the multicast group. Only other processes that are part of the same group will receive each others messages. The multicast group ip address doesn't conflict with normal unicast ip addresses since they have a specific range that is excluded from normal unicast usage. 224.0.0.0 to 239.255.255.255 (inclusive)
defaults:224.2.2.3. The address 224.0.0.0 is reserved and should not be used.

multicast-port: The port of the multicast socket where the Hazelcast Node listens and sends discovery messages on. Unlike normal unicast sockets where only a single process can listen to a port, with multicast sockets multiple processes can listen to the same port. So you don't need to be worried about having multiple Hazelcast nodes that run on the same JVM are going to conflict.
defaults:54327

TODO: Why are we talking about time to live in seconds? TTL on multicast isn't interpreted in seconds.
multicast-time-to-live-seconds:
defaults:32
/**
   * Set the default time-to-live for multicast packets sent out
   * on this <code>MulticastSocket</code> in order to control the 
   * scope of the multicasts.
   *
   * <P> The ttl <B>must</B> be in the range <code> 0 <= ttl <=
   * 255</code>


multicast-timeout-seconds:
defaults:2

trusted-interfaces:
defaults:


\subsection{Configuring Hazelcast for full TCP/IP cluster}

Clouds often don't provide multicast.

The idea is that there should be one or more well known machines.

It can be configured through setting the hostname or can be configured through the interface. 

If multicast is not preferred way of discovery for your environment, then you can configure Hazelcast for full TCP/IP cluster. As configuration below shows, while enable attribute of multicast is set to false, tcp-ip has to be set to true. For the none-multicast option, all or subset of cluster members' hostnames and/or ip addresses must be listed. Note that all of the cluster members don't have to be listed there but at least one of them has to be active in cluster when a new member joins. The tcp-ip tag accepts an attribute called "conn-timeout-seconds". The default value is 5. Increasing this value is recommended if you have many IP's listed and members can not properly build up the cluster.

\begin{lstlisting}[language=xml]
<hazelcast>
    ...
    <network>
        <port auto-increment="true">5701</port>
        <join>
            //why are the multicast-group and port still set?
            <multicast enabled="false"/>
            <tcp-ip enabled="true">
                <hostname>machine1</hostname>
                <hostname>machine2</hostname>
                <hostname>machine3:5799</hostname>
                <interface>192.168.1.0-7</interface>     
                <interface>192.168.1.21</interface> 
            </tcp-ip>
        </join>
        ...
    </network>
    ...
</hazelcast>
\end{lstlisting}

<required-member></required-member>: defaults to 127.0.0.1


\subsection{Configuring Hazelcast for EC2 Auto Discovery}

Hazelcast supports EC2 Auto Discovery as of 1.9.4. It is useful when you don't want or can't provide the list of possible IP addresses. Here is a sample configuration: Disable join over multicast and tcp/ip and enable aws. Also provide the credentials. The aws tag accepts an attribute called "conn-timeout-seconds". The default value is 5. Increasing this value is recommended if you have many IP's listed and members can not properly build up the cluster.

\begin{lstlisting}[language=xml]
<hazelcast>
    ...
    <network>
       <join>
           <multicast enabled="false"/>
           <tcp-ip enabled="false">
              todo: why is interface set?
              <interface>192.168.1.2</interface>
            </tcp-ip>
    
            <aws enabled="true">
               <access-key>my-access-key</access-key>
               <secret-key>my-secret-key</secret-key>
               <region>us-west-1</region>                              <!-- optional, default is us-east-1 -->
               <security-group-name>hazelcast-sg</security-group-name> <!-- optional -->
               <tag-key>type</tag-key>                                  <!-- optional -->
               <tag-value>hz-nodes</tag-value>                          <!-- optional -->
            </aws>
       </join>
    </network>
</hazelcast>
\end{lstlisting}

\section{Creating Separate Clusters}

By specifying group-name and group-password, you can separate your clusters in a simple way; dev group, production group, test group, app-a group etc... 

todo: password is optional, it defaults to 'drowssap'

\begin{verbatim}
<hazelcast>
    <group>
        <name>dev</name>
        <password>dev-pass</password>
    </group>
    ...
</hazelcast>
\end{verbatim}
You can also set the groupName with Config API. JVM can host multiple Hazelcast instances (nodes). Each node can only participate in one group and it only joins to its own group, does not mess with others. Following code creates 3 separate Hazelcast nodes, h1 belongs to app1 cluster, while h2 and h3 are belong to app2 cluster.

\begin{verbatim}
Config configApp1 = new Config();
configApp1.getGroupConfig().setName("app1");

Config configApp2 = new Config();
configApp2.getGroupConfig().setName("app2");

HazelcastInstance h1 = Hazelcast.newHazelcastInstance(configApp1);
HazelcastInstance h2 = Hazelcast.newHazelcastInstance(configApp2);
HazelcastInstance h3 = Hazelcast.newHazelcastInstance(configApp2);
\end{verbatim}

\section{Specifying network interfaces}

You can also specify which network interfaces that Hazelcast should use. Servers mostly have more than one network interface so you may want to list the valid IPs. Range characters ('*' and '-') can be used for simplicity. So 10.3.10.*, for instance, refers to IPs between 10.3.10.0 and 10.3.10.255. Interface 10.3.10.4-18 refers to IPs between 10.3.10.4 and 10.3.10.18 (4 and 18 included). If network interface configuration is enabled (disabled by default) and if Hazelcast cannot find an matching interface, then it will print a message on console and won't start on that node.

\begin{lstlisting}[language=xml]
<hazelcast>
    ...
    <network>
        ....
        <interfaces enabled="true">
            <interface>10.3.16.*</interface> 
            <interface>10.3.10.4-18</interface> 
            <interface>192.168.1.3</interface>         
        </interfaces>    
    </network>
    ...
</hazelcast>
\end{lstlisting}

\section{Partition Group Configuration}
Hazelcast distributes key objects into partitions (blocks) using a consistent hashing algorithm and those partitions are assigned to nodes. That means an entry is stored in a node which is owner of partition to that entry's key is assigned. Number of total partitions is default 271 and can be changed with configuration property hazelcast.map.partition.count. Along with those partitions, there are also copies of them as backups. Backup partitions can have multiple copies due to backup count defined in configuration, such as first backup partition, second backup partition etc. As a rule, a node can not hold more than one copy of a partition (ownership or backup). By default Hazelcast distributes partitions and their backup copies randomly and equally among cluster nodes assuming all nodes in the cluster are identical.

What if some nodes share same JVM or physical machine or chassis and you want backups of these nodes to be assigned to nodes in another machine or chassis? What if processing or memory capacities of some nodes are different and you do not want equal number of partitions to be assigned to all nodes?

You can group nodes in same JVM (or physical machine) or nodes located in the same chassis. Or you can group nodes to create identical capacity. We call these groups as partition groups. This way partitions are assigned to those partition groups instead of single nodes. And backups of these partitions are located in another partition group.

When you enable partition grouping, Hazelcast presents two choices to configure partition groups at the moments.

First one is to group nodes automatically using IP addresses of nodes, so nodes sharing same network interface will be grouped together.

\begin{verbatim}
<partition-group enabled="true" group-type="HOST_AWARE" />
Config config = ...;
PartitionGroupConfig partitionGroupConfig = config.getPartitionGroupConfig();
partitionGroupConfig.setEnabled(true).setGroupType(MemberGroupType.HOST_AWARE);
\end{verbatim}

Second one is custom grouping using Hazelcast's interface matching configuration. This way, you can add different and multiple interfaces to a group. You can also use wildcards in interface addresses.

\begin{verbatim}
<partition-group enabled="true" group-type="CUSTOM">
    <member-group>
        <interface>10.10.0.*</interface>
        <interface>10.10.3.*</interface>
        <interface>10.10.5.*</interface>
    </member-group>
    <member-group>
        <interface>10.10.10.10-100</interface>
        <interface>10.10.1.*</interface>
        <interface>10.10.2.*</interface>
    </member-group
</partition-group>
Config config = ...;
PartitionGroupConfig partitionGroupConfig = config.getPartitionGroupConfig();
partitionGroupConfig.setEnabled(true).setGroupType(MemberGroupType.CUSTOM);

MemberGroupConfig memberGroupConfig = new MemberGroupConfig();
memberGroupConfig.addInterface("10.10.0.*")
    .addInterface("10.10.3.*").addInterface("10.10.5.*");

MemberGroupConfig memberGroupConfig2 = new MemberGroupConfig();
memberGroupConfig2.addInterface("10.10.10.10-100")
    .addInterface("10.10.1.*").addInterface("10.10.2.*");

partitionGroupConfig.addMemberGroupConfig(memberGroupConfig);
partitionGroupConfig.addMemberGroupConfig(memberGroupConfig2);
\end{verbatim}

\section{Network Partitioning (Split-Brain Syndrome)}

\section{What is next}
Lorem ipsum dolor sit amet, consectetur adipiscing elit. Morbi libero sem,
interdum eget varius vel, faucibus placerat purus. Sed vulputate diam sit amet
risus dapibus dignissim. Praesent lobortis eleifend augue. Cum sociis natoque
penatibus et magnis dis parturient montes, nascetur ridiculus mus. Morbi libero
turpis, viverra ac vulputate a, faucibus vel quam. Quisque interdum congue
lacus, in tempus nisl tincidunt at. Curabitur sed eros eu enim vehicula
fermentum quis nec justo. Vestibulum rutrum laoreet est, eget condimentum justo
feugiat at. Cras ac sem ac magna ornare tempor non nec nisl. Maecenas feugiat
fringilla nisl, vitae ullamcorper ante posuere a. Sed mollis lacinia interdum.
Vivamus vel urna metus. Nulla eget tellus sem. Praesent volutpat suscipit nulla,
nec dictum arcu iaculis id. Duis pharetra vestibulum sapien, quis pulvinar odio
pharetra id. Cras at erat velit, vel tincidunt elit. Curabitur vehicula leo eu
odio vulputate ac consequat nulla ultricies. Maecenas venenatis condimentum
urna ut ultrices. Aliquam blandit fermentum eros, ac lacinia sem scelerisque
at. Nullam vitae nisi at erat posuere cursus a non velit.
